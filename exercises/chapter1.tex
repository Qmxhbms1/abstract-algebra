\chapter{Introduction to Groups}

\section{Basic Axioms and Examples}

% Problem 1.1.1 and 1.1.2
\begin{problem}
  Let $G$ be a group.
  Determine which of the following binary operations are associative (1.1.1) and which are commutative (1.1.2):
  \begin{enumerate}[label=(\alph*)]
    \item the operation $\bigstar$ on $\Z$ defined by $a \bigstar b = a - b$
    \item the operation $\bigstar$ on $\R$ defined by $a \bigstar b = a + b + ab$
    \item the operation $\bigstar$ on $\Q$ defined by $a \bigstar b = \frac{a + b}{5}$
    \item the operation $\bigstar$ on $\Z \times \Z$ defined by $(a, b) \bigstar (c, d) = (ad + bc, bd)$
    \item the operation $\bigstar$ on $\Q - \{0\}$ defined by $a \bigstar b = \frac{a}{b}$
  \end{enumerate}
\end{problem}

\begin{solution}
  \begin{enumerate}[label=(\alph*)]
    \item $(a - b) - c = a - (b - c)$ implies $c = 0$, but since $c$ is arbitrary from $\Z$ this operation is not associative. It is also clearly not commutative.
    \item $(a + b + ab) \bigstar c = a \bigstar (b + c + bc)$ gives us $ab + ac + bc + abc = bc + ab + ac + abc$, thus this operation is associative. It is commutative by the commutativity of the addition of real numbers.
    \item Solving $\frac{\frac{a + b}{5} + c}{5} = \frac{a + \frac{b + c}{5}}{5}$, we get $4c = a$, thus the operation is not associative in all cases. It is commutative by the commutativity of addition.
    \item This operation is precisely addition defined on $\Q$, which is associative and commutative.
    \item We get $\frac{a}{bc} = \frac{ac}{b}$, which need not hold in $\Q - \{0\}$. It is also not commutative.
  \end{enumerate}
\end{problem}

% Problem 1.1.12
\setcounter{problem}{11}
\begin{problem}
  Find the order of the following elements of the additive group $(\Z / 12 \Z)^{\times}$: $\bar{1}, \bar{-1}, \bar{5}, \bar{7}, \bar{-7}, \bar{13}$.
\end{problem}

\begin{solution}
  By definition we know $|\bar{1}| = 1$.
  We also easily get the following:
  \begin{itemize}
    \item $\bar{-1} \cdot \bar{-1} = \bar{1}$, i.e., $\bar{-1}$ has order $2$.
    \item $\bar{5} \cdot \bar{5} = \bar{25} = \bar{1}$, thus $\bar{5}$ has order $2$.
    \item $\bar{7} \cdot \bar{7} = \bar{49} = \bar{1}$, and $\bar{7}$ has order $2$.
    \item $\bar{-7} \cdot \bar{-7} = \bar{49} = \bar{1}$, hence $\bar{-7}$ has order $2$.
    \item $\bar{13} = \bar{1}$, which has order $1$.
  \end{itemize}
\end{problem}

% Problem 1.1.15
\setcounter{problem}{14}
\begin{problem}
  Prove that $(a_1 a_2 \ldots a_n)^{-1} = a_n^{-1} a_{n-1}^{-1} \ldots a_1^{-1}$ for all $a_1 \ldots a_n \in G$.
\end{problem}

\begin{solution}
  We have
  \[(a_1 a_2 \ldots a_n) (a_1 a_2 \ldots a_n)^{-1} = 1.\]
  Thus we get
  \[a_2 \ldots (a_1 a_2 \ldots a_n)^{-1} = a_1^{-1}.\]
  Continuing like this we arrive at our desired result.
\end{solution}

% Problem 1.1.16
\begin{problem}
  Let $x$ be an element of $G$.
  Prove that $x^2 = 1$ if and only if $|x|$ is either $1$ or $2$.
\end{problem}

\begin{solution}
  By definition we know that $|x|$ is the least integer $n$ such that $x^n = 1$.
  Since we know that $x^2 = 1$ we know that $n \le 2$.
  The only integers less than or equal to $2$ are $1$ and $2$.
\end{solution}

% Problem 1.1.17
\begin{problem}
  Let $x$ be an element of $G$.
  Prove that if $|x| = n$ for some positive integer $n$, then $x^{-1} = x^{n-1}$.
\end{problem}

\begin{solution}
  Multiplying $x^n = 1$ by $x^{-1}$ and applying exercise 19 gives us the desired result.
\end{solution}

% Problem 1.1.32
\setcounter{problem}{31}
\begin{problem}
  If $x$ is an element of finite order $n$ in $G$, prove that the elements $1, x, x^2, \ldots x^n$ are all distinct.
  Deduce that $|x| \le |G|$.
\end{problem}

\begin{solution}
  For a contradiction, assume there are $n > i > j \ge 0$ such that $x^i = x^j$.
  Then $x^i x^{-j} = 1$.
  By exercise 19 we then know that $x^{i - j} = 1$.
  By definition we have that $i - j > 0$ and $n > i - j$, contradicting that $n$ is the minimality of $n$.
  This implies that if we have an element $x$ of order $n$ than there exist at least $n$ distinct elements, i.e., $|G| \ge n = |x|$.
\end{solution}

% Problem 1.1.33
\begin{problem}
  Let $x$ be an element of finite order $n$ in $G$.
  \begin{enumerate}[label=(\alph*)]
    \item Prove that if $n$ is odd then $x^i \neq x^{-i}$ for all $i = 1, 2 \ldots n - 1$.
    \item Prove that if $n = 2k$ and $1 \le i < n$ then $x^i = x^{-i}$ if and only if $i = k$.
  \end{enumerate}
\end{problem}

\begin{solution}
  \begin{enumerate}[label=(\alph*)]
    \item Let $n = 2k + 1$ and $x^i = x^{-i}$.
      Multiplying both sides by $x^{i}$ we get $x^{2i} = 1$.
      This implies that $2i > n$.
      However, then we have $x^{2i - n} = 1$, which also implies that $2i - n > n$, i.e., $i > n$, a contradiction.
    \item Let $n = 2k$.
      It is simple to see that if $i = k$ then $x^i = x^{-i}$.
      If we have $x^{i} = x^{-i}$ then $x^{2i} = 1$, which implies that $i > k$ as above.
      Then we again take $x^{2i - n} = 1$ and find that $i = k$.
  \end{enumerate}
\end{solution}


